\section{Repositorio remoto: GitHub}

% Funcionamento e uso de un CVS: Git
\title[Git e GitHub]{Repositorio remoto: GitHub}
\author[Fran Rúa e Breixo Camiña]{}

\begin{frame}
  \titlepage
  \begin{figure}[H]
    \centering
    \label{fig:github}
    \includegraphics[scale=.4]{github}
  \end{figure}
\end{frame}

\begin{frame}[fragile]
  Existen varios repositorios remotos, como poden ser \textbf{Bitbucket de Atlassian},
  \textbf{GitLab de GitLab Inc.}, ... Pero o máis extendido e o que conta coa maior
  comunidade é GitHub. GitHub conta con milleiros de proxectos, dende o código
  que estamos a subir nós agora ata o código fonte do kernel de Linux.
  \begin{multicols}{2}
    \begin{figure}
      \includegraphics[scale=0.1]{bitbucket}
    \end{figure}
    \columnbreak
    \begin{figure}
      \includegraphics[scale=0.1]{gitlab}
    \end{figure}
  \end{multicols}
\end{frame}

\begin{frame}[fragile]
  \frametitle{Repositorio remoto}
  \scriptsize
  Para crear o noso repositorio remoto, antes crearemos unha conta en GitHub. Logo de rexistrarnos, creamos un repositorio novo (New repository). Logo de indicarlle o nome, seguiremos os pasos que nos indican. Como xa temos o noso repositorio creado, con commits xa feitos sobre ficheiros, simplemente executaremos \textbf{git remote add origin https://github.com/BreixoCF/charlagit.git} e posteriormente faremos \textbf{git push -u origin master}.
  \tiny
\begin{verbatim}
	breixocf@BreixoCF ~/Desktop/CharlaGit 19 $ git remote add origin 
	https://github.com/BreixoCF/charlagit.git
	breixocf@BreixoCF ~/Desktop/CharlaGit 20 $ git push -u origin master 
	Username for 'https://github.com': breixocf
	Password for 'https://breixocf@github.com': 
	Counting objects: 13, done.
	Delta compression using up to 4 threads.
	Compressing objects: 100% (9/9), done.
	Writing objects: 100% (13/13), 1.14 KiB | 0 bytes/s, done.
	Total 13 (delta 2), reused 0 (delta 0)
	To https://github.com/BreixoCF/charlagit.git
	 * [new branch]      master -> master
	Branch master set up to track remote branch master from origin.
\end{verbatim}	
\end{frame}

\begin{frame}[fragile]
  \frametitle{Octavo comando: git clone}
  \begin{block}{git clone}
    O comando git clone serve para clonar un repositorio remoto a local, executando tres pasos nun solo: git init, git remote add e git pull.
  \end{block}
  Coñecendo este comando, poderíamos executar cunha sola sentenza o clonado de
  repositorio da seguinte forma: \textbf{git clone
    https://github.com/BreixoCF/charlagit.git}.
    \tiny
	\begin{verbatim}
	~ % git clone https://github.com/BreixoCF/charlagit.git
	Cloning into 'charlagit'...
	remote: Counting objects: 16, done.
	remote: Compressing objects: 100% (9/9), done.
	remote: Total 16 (delta 3), reused 16 (delta 3), pack-reused 0
	Unpacking objects: 100% (16/16), done.
	Checking connectivity... done.
	\end{verbatim}
\end{frame}

\begin{frame}[fragile]
  \scriptsize
  Agora copiaremos este código dunha pila en Python nun ficheiro que chamaremos \textbf{stack.py}:
  \tiny
\begin{verbatim}
	# Implementación de unha pila facendo uso de programacion orientada a obxectos.
	class Stack:
	    def __init__(self):
	        self.items = []
	
	    def vacia(self):
	        return self.items == []
	
	    def apilar(self, item):
	        self.items.append(item)
	
	    def cima(self):
	        if self.vacia():
	            return []
	        return self.items[-1]
	
	    def desapilar(self):
	        del self.items[-1]
	
	    def tam(self):
	        return len(self.items)
\end{verbatim}
\end{frame}

\begin{frame}[fragile]
  \normalsize
  Agora imos a subir o ficheiro ao repositorio de \textbf{GitHub} que acabamos de crear. Para eso, engadimos o ficheiro \textit{stack.py} á zona de preparación (Index) e posteriormente ao repositorio local (HEAD).
  \tiny 
  	\begin{verbatim}
  		breixocf@BreixoCF ~/Desktop/CharlaGit 21 $ git add stack.py 
  		breixocf@BreixoCF ~/Desktop/CharlaGit 22 $ git commit -m "Engadida a pila OO stack.py"
  		[master 111365e] Engadida a pila OO stack.py
  		 1 file changed, 21 insertions(+)
  		 create mode 100644 stack.py
  	\end{verbatim}
  \large
  \begin{center}
    \textbf{¿Xa estaría subido ao repositorio remoto?}\\
    \vspace{1cm}
    \textbf{\Huge NON}
  \end{center}
\end{frame}

\begin{frame}[fragile]
  \frametitle{Noveno comando: git push}
  \begin{block}{git push}
    Este comando serve para subir os cambios do repositorio local (HEAD) ao repositorio remoto.
  \end{block}
  \scriptsize
  Se executamos \textbf{git push origin master}, podremos comprobar na web de github que o ficheiro está engadido correctamente.
  \tiny 
	\begin{verbatim}
		breixocf@BreixoCF ~/Desktop/CharlaGit 21 $ git add stack.py 
		breixocf@BreixoCF ~/Desktop/CharlaGit 22 $ git commit -m "Engadida a pila OO stack.py"
		[master 111365e] Engadida a pila OO stack.py
		 1 file changed, 21 insertions(+)
		 create mode 100644 stack.py
		breixocf@BreixoCF ~/Desktop/CharlaGit 23 $ git push origin master 
		Username for 'https://github.com': breixocf
		Password for 'https://breixocf@github.com': 
		Counting objects: 4, done.
		Delta compression using up to 4 threads.
		Compressing objects: 100% (3/3), done.
		Writing objects: 100% (3/3), 489 bytes | 0 bytes/s, done.
		Total 3 (delta 1), reused 0 (delta 0)
		To https://github.com/BreixoCF/charlagit.git
		   621a67b..111365e  master -> master
	\end{verbatim}
\end{frame}

\begin{frame}[fragile]
  \frametitle{Décimo comando: git fetch}
  \begin{block}{git fetch}
    Este comando serve para actualizar o repositorio local respecto o repositorio remoto indicado.
  \end{block}
  \tiny
\begin{verbatim}
	breixocf@BreixoCF ~/Desktop/CharlaGit 27 $ git fetch origin master
	From https://github.com/BreixoCF/charlagit
	 * branch            master     -> FETCH_HEAD
\end{verbatim}
	\normalsize
  Como non hai cambios nos ficheiros non pasaría nada.
\end{frame}


\begin{frame}[fragile]
  \frametitle{Undécimo comando: git pull}
  \begin{block}{git pull}
    Este comando serve para descargar o commit máis novo do repositorio remoto á nosa zona de traballo (workspace).
  \end{block}
  \scriptsize
  Se estivéramos dous ou máis usuarios traballando sobre o mismo repositorio remoto, poderíamos descargarnos os cambios que fixo o outro usuario executando \textbf{git pull origin master}.
  \tiny 
	\begin{verbatim}
	breixocf@BreixoCF ~/Desktop/CharlaGit 24 $ git pull origin master 
	From https://github.com/BreixoCF/charlagit
 	* branch            master     -> FETCH_HEAD
	Already up-to-date.
	\end{verbatim}
	\normalsize
  Como non hai cambios nos ficheiros non pasaría nada.
\end{frame}

\begin{frame}
\Huge
	\begin{center}
	Ata aquí por hoxe, pero para a semana máis e mellor!!
	\end{center}
\end{frame}

